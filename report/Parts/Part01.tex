\chapter{Introduction}
	
	The purpose of this project was to implement, a real time and cross platform
	ray marcher written in GLSL that can exploit the power of stream
	processing provided by the GPU. To achieve this it we implemented the ray marching
    algorithm in a GLSL fragment shader that renders the scene by doing per-pixel
	computation.
	
	The idea included the possibility to control the rendering parameters in real time
	using a dynamic compositing system which interface would also be rendered through OpenGL.
	
	From there the project advanced and we implemented deffered shading with multiple
    render targets on the ray marcher in order to add a set of post effect shaders. 
    We added 
	
    since the integration of a node compositor took longer time than expected,
    we decided to use the Qt framework to speed up the development of the user interface.
    This way we would be able to use widgets such as sliders and color pickers,
    and also have text rendering that we would probably not have time to implement
    in OpenGL. It turned out that mixing
    Qt widgets and OpenGL rendering in the same frame is not as well supported as
    we hoped. Therefore we dropped implementing the widgets in OpenGL in order to
    be able to finish the project and concentrate on more important features.

    The entire rendering pipeline is implemented in OpenGL, but not the user interface
    of the compositing system (rendered separately using the Qt graphics API).

    Here is a quick summary of the interesting features of this project:
    \begin{itemize}
        \item \textbf{Real-time} ray marching implemented on the GPU through a GLSL fragment shader. \footnote{See part \ref{chap:raymarching}.}
        \item Post effects also implemented using shaders\footnote{See part \ref{chap:postfx}.}:
        \begin{itemize}
            \item Radia blur
            \item Bloom
            \item Edge detection
            \item Sepia and black and white color filters
            \item Depth of field with bokehs
        \end{itemize}
        \item Deferred shading with multiple target rendering\footnote{See part \ref{chap:compositor}.}. 
        \item A completely dynamic and modular compositing system\footnote{See part \ref{chap:compositor}.}. This feature is
        important because it forced us to rethink the way we use the OpenGL API
        in order to integrate it in a dynamic visual scripting system and make it
        possible to change the rendering pipeline at run-time. We also wrapped
        most of the use of OpenGL into classes (Shader, FrameBuffer, Texture...).
    \end{itemize}

\vspace{1cm}

The entire source code for this project is available online at \url{http://github.com/nical/GLSL-Raymarching}.
This project uses th follownig exetral libraries:
\begin{itemize}
    \item OpenGL version 3.3+
    \item Glew
    \item Qt 4.7
    \item Kiwi (backend of the compositing system, developed as a side project by Nicolas Silva - s111781)
\end{itemize}
