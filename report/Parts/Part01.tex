\chapter{Introduction}
	
	The purpose of this project was to implement, at first, a real time
	raytracer written in GLSL so that it could exploit the power of stream
	processing given by the GPU. To achieve this it was needed to implement
	a GLSL fragment shader that rendered the scene by doing per-pixel
	computation.
	
	The idea included the possibility to control the rendering in real time
	using an OpenGL rendered UI connected to the shader to, for example,
	change the colour of the sky or of the rendered objects.
	
	From there the project advanced and included a set of post effect shaders,
	implemented as deffered shading shaders, and a real time node compositor
	to create and modify in real time the rendering chain. In this way
	is possible, for example, to change the post effects order, change the colours of the
	elements or modify parameters of the post effects.
	
	OpenGL, however, has some limitations for UI design and rendering, since
	even trying to draw a simple string on the screen requires the developers
	to implement specific geometry and fragment shaders.
	
	To tackle this problem it QT was chosen as framework for the creation
	of the User Interface. In this way it was possible to render the scene
	in a window and then add the node compositor beneath it. 
