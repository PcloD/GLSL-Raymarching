\chapter{Raymarching \label{chap:raymarching} }
This part is mostly relevant to the Rendering course.
~\\
Ray marching is a technic similar to ray tracing on many aspects. The colour of
each fragment is determined by using rays going from the viewer to the scene with
a pinhole camera model. Rays are not casted directly to the surfaces of the objects
like in most of the methods derived from ray tracing.

Instead the distance to the closest object is computed and the ray is advanced one step towards
its vector by this distance. This procedure is repeated
until the ray reaches an object or a maximum number of steps is reached. Another version can limit
the distance marched. The step version, however, is more compatible with the version
of GLSL that comes together with WebGL (a slightly modified version of OpenGL ES2), which requires
loops to be limited in the number of steps. 

This method can seem sub-optimal at first, as every ray need several iterations before reaching an object,
but it turns out that it is often faster than raytracing for the following reasons:
\begin{itemize}
 \item most simple objects have a very fast distance equation\footnote{As can be seen on this
 	very useful web page: http://www.iquilezles.org/www/articles/distfunctions/distfunctions.htm}.
 \item distance functions can be modified in order to add interesting effects such as
	infinite repetition for a negligible cost.
 \item it provides local and global informations, which makes possible additional effects
 	such as Ambient Occlusion or Soft Shadows with no to little extra effort
 \item it requres less branching. Since GPUs are not meant for conditional execution, then
 	the less branches executed on the GPU the better, since fragment shaders run SIMD
 	(Single Instruction Multiple Data) branching.
\end{itemize}

\image{Images/RaymarchedScene.png}{Example of a raymarched scene with ambient occlusion and soft shadows.}{0.8}{img:RaymarchingExample}

\section{Infinite repetition} 

In raymarching repeating objects infinitely is both easy and cheap in term of GPU cycles.

Let's take a look at the distance function of a sphere.

\begin{lstlisting}
#define center vec3(0,0,0)
#define radius 2.0
float SphereDistance( vec3 point )
{
    return length( center - point ) - radius;
}
\end{lstlisting}

This functions returns the distance between a point and the center of the sphere
minus the radius of the sphere.

To repeat infinitely the sphere arround the X axis we only need to add a modulo
operation like follows:

\begin{lstlisting}
#define dist 10
// ...
point.x = mod( point.x + dist/2, dist ) - dist/2;
return length( center - point ) - radius;
}
\end{lstlisting}

An infinte number of copies of the same object will appear and the frame rate will not change
noticeably. Adding $\frac{dist}{2}$ inside the modulo and removing it afterward does not alter
the position of the object but is necessary because, without it, the sphere would
be cut by the modulo as it is centered on zero.

%todo: image of a cut sphere

%todo: scheme to explain the effect of modulo on the position of the distance field

This is one of the big advantages of ray marching. Repeating geometry to infinity
at almost no cost is a nice feature, since it can be used for a high number of objects,
such as grass or honeycombs. However, the resulting image could be somewhat boring because
of the repetition itself. This is not a big problem however because we can modify
the distance function in order to modify the shape in function of the position.
For instance, to have variable sphere radius on the same object wa can add the following
line before applying the modulo:

\begin{lstlisting}
float newRadius =  1+2*sin(floor(point.x/dist)));
\end{lstlisting}

Here, \syntax{floor(point.x/dist)} returns a different constant number for replication
of the sphere as it is divided by the distance between two repetitions, and the decimal part
is removed. This number grows along with the x position of the sphere, so it
can be interesting to place it inside a periodic function to clamp it
whithin a certain range. Then it is just a matter of tweaking the mathematical
function in order to break periodicity and introduce a feeling of randomness.

To sum up, with the following distance function, we obtain an infinite number
of spheres which radius varies between $1.0$ and $3.0$. 

\begin{lstlisting}
#define dist 10
#define center vec3(0,0,0)
float SphereDistance( vec3 point )
{
    float newRadius =  1+2*sin(floor(point.x/dist)));
    point.x = mod( point.x + dist/2, dist ) - dist/2;
    return length( center - point ) - newRadius;
}
\end{lstlisting}

Performance-wise, this is a great feature. An infinity of objects in the screen introduced by
one simple three lines distance function that contains no loop, and not even a branch.  


\section{Computing normals}

At first we tried to compute normals using a specific function for each kind of
shape. This is simple for a sphere but can get complex for other objects.
In fact there is a much simpler and efficient way to compute normals, by using
the expression of the gradient:
Given a point \syntax{p} on the surface of an object
described by the distance function \syntax{f}, its normal \syntax{n} can be evaluated using a very
small \syntax{epsilon} quantity.

This can be expressed in GLSL code in the following way:

\begin{lstlisting}
n = normalize( vec3(
      f(p + vec3(epsilon,0,0)  ) - f(p - vec3(epsilon,0,0))
    , f(p + vec3(0,epsilon,0)  ) - f(p - vec3(0,epsilon,0))
    , f(p + vec3(0,0,epsilon)  ) - f(p - vec3(0,0,epsilon))
) );
\end{lstlisting}

This is very convenient on objects more complicated than a sphere, since
the normal vector is used in various occasions, such as reflections or Ambient
Occlusion Computations.

\section{Ambient occlusion}

Ambient occlusion (AO) is the effect of small areas being darkened by taking into
account the attenuation of light due to occlusion by surrounding objects.
AO is a crude approximateion of the way light bounces around in real world, in
order to provide with more realistic results. It does not give as good results as
global illumination techniques like radiosity, but helps providing convincing graphics
in real time.

Thanks to Ray Marching technique the Ambient Occlusion effect can be computed easily.
The idea is to march a new ray for a fixed number of same length steps along the normal
of the surface that was hit and use the distance to the closest surface at each step
to compute the occlusion effect due to how near that is. This value is then used to
apply a shadow effect on the pixel that is currently computed. The effect of a surface is
divided by the square of the distance, as occlusions are less important if they are further
away from the pixel.

\begin{lstlisting}
float AmbientOcclusion (vec3 point, vec3 normal, float stepDistance, float samples) {
	float occlusion;
	int tempMaterial;   //  Needed by the DistanceField function
	for (occlusion = 1.0 ; samples > 0.0 ; samples--) {
		occlusion -= (samples * stepDistance -
				(DistanceField( point + normal * samples * stepDistance, tempMaterial))
			) / pow(2.0, samples);
	}
	return occlusion;
}
\end{lstlisting}

This computation can be done efficiently thanks to Raymarching since it requires informations
on the distance of objects along the ray.

\section{Soft Shadows}

Rendering shadows in Raymarching is based on a simple idea: once a ray has hit something
other rays are casted from that position towards the lights in the scene (one ray
per light). If the ray manages to reach the light then it means that that point is
illuminated by that source.

Using this data is possible to generate hard shadows, since it possible to know either if
the ray hitted the light or not, starting from a specific point. To generate
soft shadows it would be necessary to cast more than one ray per light by jittering the starting
position. This greatly increases the computation cycles required per pixel.

With Raymarching, however, each ray can be raymarched to retrieve informations about the whereabouts 
of the objects along its path. In this way it is possible to know how near the ray
passed from an object and use that data to generate soft shadow edges by using the distance
marched and how near to an object the ray passes.

\label{lst:SoftShadow}\begin{lstlisting}
for( float t = (mint + rand(gl_FragCoord.xy) * 0.01); t < maxt; )
{
    float nextDist = min(
        BuildingsDistance(landPoint + lightVector * t )
        , RedDistance(landPoint + lightVector * t )
    );

    if( nextDist < 0.001 ){ return 0.0; }
    
    penumbraFactor = min( penumbraFactor, iterations * nextDist / t );
    t += nextDist;
}
\end{lstlisting}

The \syntax{iterations} variable is used to modify the softness of the shadow. The lower the
value is, the softer the shadow is.
