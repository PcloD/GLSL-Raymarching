\chapter{The compositing system\label{chap:compositor}}

This part of the report is mainly related to problematics of the Computer graphics course.

\section{Node compositing}

Node based compositors are widely used in computer graphics softwares. They provide
an efficient way to compose a rendering pipeline at runtime in a way that is accessible
to non-programmers (such as graphic artists). In our case it provides us with a
nice framework to experiment with the different parts of the rendering pipeline.

Our system allowed us to very quckly add new shaders and play with their parameters
and composition within the application instead of doing so in the code. 

Our node compositor is constituted of "boxes", the nodes, that can be connected
through their inpout and output ports. Ports are typed to ensure correct rooting.
To decide the order of processing of the nodes, we added a special node,
the screen node, that takes a texture in parameter and render it to the screen.
the dependency chain is then computed recursively starting from this node. All
the other shader based nodes - the ray marcher and the post effects - cannot render
to the screen. Instead, they render into textures grouped together into frame
buffer objects.

\section{The backend of the compositor}

The backend of the compositor is an open source external library named Kiwi, developed as a side
project, and available at \url{http://github.com/nical/kiwi}.

We used the following features provided by Kiwi:
\begin{itemize}
    \item Data structures for nodes, ports and pipelines with facilities to traverse the node graph.
    \item Generic runtime typing system allowing to create types - in our case we added
    types such as \textit{Texture}, \textit{Float}, \textit{Color}, etc.
\end{itemize}

\section{The frontend}

The user interface was implemented for this project, using the Qt framework. Because
of time constraint we only implemented basic interactions with the system, so the
user experience is not optimal (connecting/deconnecting ports could be operated
through drag and drop, for instance), yet it provides all the features we wanted for
this project.

We initially planned to implement the user interface display in OpenGL on top of
the rendered scene. It turned out that Qt does not make this kind of things easy.
If we wanted to use widgets provided by Qt, the scene had to be rendered as a
backround of a QGraphicsScene and the nodes displayed using the QGraphicsItem class.
This mixed use of Qt and OpenGL is not yet fully supported, and even the only tutorial
available on the subject from the Qt documentation compiled but did not work properly.
This is partly due to the way Qt handles OpenGL contexts. When experimenting with
it and looking inside a GPU debugger, we found out that Qt was creating four different
OpenGL contexts, and that context switching did not happen correctly. With the way
context were handled, we also had errors when trying to use the GLEW library to
initialize the OpenGL extensions we needed in cross platform way.

Finally we decided to separate the render frame and the compositor's user interface
and only use the standard Qt API to display the node compositor, while the rendering
pipeline would still be implemented in OpenGL, in a separate widget. This way we
stopped having OpenGL context problems, and we could use sliders and color pickers
with the compositing system.

\section{From hard-coded to dynamic rendering pipeline}

Most of the work of integrating the node compositor - if we don't consider the
time spent on resolving problems with Qt and OpenGL contexts - consisted in redesigning
our application in a way that made it possible to reorganize parts of the
rendering process at runtime.

In the first iterations of the project, the rendering pipeline was basically implemented
in two functions, the initialization function and the display function. We quickly
set up a base to display the ray marching shader by rendering it into a quad that
fills the screen. We then splitted the rendering process so that the ray marching
shader would render into a frame buffer object containing two textures - one for
the rendered scene, and another for additionnal per fragment informations (normals and depth)
that we wanted to reuse in post effects - and a second pass would use it to render
the scene to the screen with a shader based post effect. At this point every
resource was initialized together in the init function and the whole rendering
process was done in the display function. Although it made it possible for us to
rapidly get to work on the ray marcher, a complete redesign would be necessary in
order to add the compositing system we planned for. The first step was to change
the management of OpenGL resources so that we would not have to copy-paste ten lines
of code every time we needed a new texture, shader or frame buffer. We encapsulated
most of the openGL objects into classes that manage their initialization. This part
was important because it forced us to really understand how OpenGL works in order
to reorganize adequately its functionnalities in different classes. In particular,
we needed to make it possible to root any output texture port of any node into any input
texture port which requires to be carefull about how multiple target rendering works
and how frame buffers and textures are to be managed.

xxx TODO xxx

In the end, adding a new shader based effect to the program only requires to write
the shader in a separate file and to invoke the following lines of C++ code:
\begin{lstlisting}
    // initialization
    string vs, fs;
    utils::LoadTextFile("shaders/SecondPass.vert", vs );
    utils::LoadTextFile("shaders/myNewShader.frag", fs );
    auto myShader = new renderer::Shader;
    // the following info will be used to generate the node's ports and
    // to check that uniforms are passed correctly.
    renderer::Shader::LocationMap locations = {
        {"inputImage",   { Shader::UNIFORM | Shader::TEXTURE2D} },
        {"alpha",        { Shader::UNIFORM | Shader::FLOAT} },
        {"windowSize",   { Shader::UNIFORM | Shader::FLOAT2} },
        {"outputImage",  { Shader::OUTPUT  | Shader::TEXTURE2D} }
    };
    myShader->build(vs,fs,locations);
    nodes::RegisterPostFxNode( myShader  ,"Name of the effect"); // register a node type

    // create an instance of the node type (the node can be independently instanciated
    // any number of times) and allocate the necessary resources (fbo, textures...)
    kiwi::core::Node * myNode = nodes::CreatePostFxNode("Name of the effect");
    // create the GUI for the node and add it to the scene
    io::Compositor::Instance().add( new io::NodeView(QPointF(0,0), myNode) );
}
\end{lstlisting}

\image{Images/ComposingResult.png}{One of the possible outcomes using the composer}{0.8}{img:ComposingResult}

\section{Possible improvements}

xxx TODO xxx\\
implement nicer user experience\\
manage frame buffers and textures to reuse them as much as possible instead of allocating for each node\\
allow to save and restore pipelines (should be a feature of Kiwi)\\
add more nodes!\\
